\section{Konklusion}
Gruppen har endt ud med et funktionelt projekt, hvor de vigtigste features blev implementeret. Der er nogle features som ikke er blevet lavet og er derfor sat til fremtidigt arbejde. Ud fra projektet blev der valgt at disse features ikke var nødvendige for at opnå et funktionelt spil, som kunne spilles igennem og stadig fungere som et spil med fjender og genstande, som kunne samles op og anvendes.\\ Dette blev gjort da der var fokus på kommunikation mellem systemets segmenter og integrationen af disse segmenter i sidste ende. Modulerne er dog lavet med fremtidigt arbejde i tanken, så f.eks. i Game Module er det lavet relativt nemt at implementere yderligere features, så fremtidige features kunne implementeres uden at komprimere kommunikation mellem systemets segmenter.

\subsection{Personlige konklusioner}

\subsubsection{Luyen Vu}
Dette semesters projekt har været spændende og anderledes. Anderledes grundet at vi i dette semester ikke skulle arbejde med embedded software. Jeg vil argumentere for at vi i dette semester anvendte semestrets fag mere sammenlignet med de andre semestre, da vi som gruppe delte vores moduler op i de forskelige fags faggrupper. Derved har spørgsmål og troubleshooting været nemmere, idet at vi alle havde en ide om hvordan de forskellige moduler fungerede ift. en opdelling af Software- og elektronikstuderende. 
Vores produkt er endt fint ud i min optik og det har været et interessant projekt hele vejen igennem. Det faktum at det kun var software har også haft en indflydelse på mit syn af interessen af projektet.

\subsubsection{Oscar Dennis}
Projektet dette semester fungerede på en anden måde en projekter, som man har været vant til. Dette var på baggrund af at dette projekt ikke kravede samarbejde med elektronik ingeninørerne, og var derfor et rent software baseret projekt. Dette resulterede i at da vi alle var på samme studie, så var det nemmere at forstå og reviewe andres arbejde. Yderligere opstod der problemer med integration af de forskellige segmenter, da grupperne, såsom Front-end, Back-end og database, arbejde meget individuelt og opdelt. Derfor var der adskellige problemer som skulle løses før systemet kunne sættes sammen og fungere.\\
Dog har gruppen endt ud med et funktionelt projekt som jeg personligt er meget tilfreds med.

\subsubsection{Rasmus Engelund}
Dette projekt har været positivt anderledes end de tidligere, da vi denne gang skulle lave et helt softwarebaseret projekt, og ikke skulle forholde os til elektronik og embedded programmering.\\
Vi har i projektet formået at komme godt rundt i de forskellige fag hvor vi har inddraget nyttig viden, som vi har tilegnet os i de forskellige kurser. Her har projektets opbygning gjort opdeling af arbejdsopgaver nemmere, dog med undtagelse af nogle småting hvor kommunikationen og klarheden omkring arbejdsfordelingen, kunne have været bedre fastlagt.
Vi fik implementeret næsten alle de forudbestemte features, til en tilfredsstillende grad.
Generelt er vi, trods min umiddelbare tvivl om emnet, endt ud med et fornuftigt projekt, som jeg personligt er fint tilfreds med. 

\subsubsection{Magnus Blaabjerg Møller}
Igennem dette projektforløb har jeg fået et indblik i hvad det vil sige at arbejde med et rent software projekt. Der er blevet anvendt en række nye dokumentationsmetoder end på de tidligere semestre, til at dokumentere de forskellige aspekter i projektet som user stories og C4-modellen. Det har været lærigt med en ny tilgang til tingene. Jeg har gjort mig nogle gode erfaringer omkring integrationstest og test generelt af rene software projekter, blandt andet hvor vigtigt det, er at have kommunikationen igennem systemet på plads i et tidligt stadie, således at alle i gruppen er enige om hvordan kommunikationen fungerer, så der ikke opstår komplikationer under integrationstest. Derudover har jeg erfaret hvor vigtigt det er at teste tidligt og ofte hver gang der ændres i implementeringen af et modul. Hvad angår samarbejdet i gruppen har det generelt set fungeret godt, og det har været muligt at spørge hinanden om hjælp, da alle har haft en hvis viden inden for de forskellige tekniske områder igennem semestrets fag.\\

\subsubsection{Morten Høgsberg}
Dette er det første projekt uden restriktioner på projektet emne. Projektet har således for første gang
være et rent softwareprojekt. Denne gang har ideèn været at implementere et spil, hvilket er gået over
alt forventning. Arbejdsmoralen har været høj igennem projektet, med input fra alle medlemmer, og en
villighed til at lytte og give konstruktivt feedback til hinanden.
Det har været nemmere at følge up på hinandens arbejde da vi alle har haft en basis forsåelse af hvordan 
de andre har udført deres opgaver. I sidste ende har vi leveret i projekt, som opnådede at implementere
de fleste features og som fungere bedre end forventet. 

\subsubsection{Anders Hundahl}
Personligt har dette års semesterprojekt vist sig at være mere opslugende, men også mere belønnende end de tidligere projekter. Til dels tænker jeg at det skyldes at vi i år har kunnet fokusere hundrede procent på softwaren, og at vi har kunnet teste vores produkt uden at skulle sætte en masse ledninger og hardware op først, som har kunnet skabe støj eller problemer i systemet. Derudover har vi i dette semester haft mulighed og evner til at skabe et produkt som reelt set kan noget og som man kunne se virke ude i den virkelige verden. I forhold til samarbejde i gruppen føler jeg at vi har fundet ud af at arbejde sammen på en ordentlig og professionel måde, men hvor der samtidig har været mulighed og plads til at have det sjovt indimellem.

\subsubsection{Sune Dyrbye}
Dette semesterprojekt har været det bedste projekt jeg har været en del af. Samarbejdet i gruppen har været problemfrit og vi har været i stand til at færdigørevores produkt. Det har været rart at arbejde på et projekt som var rent software, da det har tilladt os at kunne teste vores implementeringer tidligt, uden at skulle vente på at hardware blev lavet. Det har været godt at kunne bruge det vi har lært i de forskellige fag i projektet. Gruppen har været godt og der har været plads til at lave sjov, mens der stadig blev arbejdet koncentretet.

\subsubsection{Jacob Hoberg}
Dette har været det første projekt, hvor det virkelig har følt som om at man havde frie tøjler. Det er resulteret i at vi har fået lavet et projekt, der har været meget sjovt at arbejde med. Samarbejdet har været godt i gruppen, da der har været stor enighed i hvad man kunne tænke sig at arbejde med, samt ambitionsniveauet har været meget ens. Desuden har der været god kemi blandt gruppemedlemmerne, hvilket har resulteret i at det ofte har været en fornøjelse at sidde og arbejde på projektet. Arbejdet har dog været meget opdelt internt i gruppen, og dette viste sig tydeligt da vi skulle integrere vores system. Så til fremtiden, skal vi være lidt bedre til at vise de ting, vi laver, frem for hinanden. 