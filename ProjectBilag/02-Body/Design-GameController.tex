\section{Game Controller Design}
Game Controllerens endelige design følger arkitekturen tæt og tager udgangspunkt i de funktioner som skal kaldes i modulet Front-end. I dette afsnit fokuseres der på de features i spillet som gør spillet funktionelt og giver brugeren et reelt gameplay.

\subsection{Map Creation}
Når bruger vælger at starte et spil eller loade et spil fra Front-end siden, skal Game Controlleren indlæse et map så Front End kan fremvise det for bruger. Det ønskes at dette sker ved at Game Controlleren indlæser et tekstdokument(.txt-fil), hvor rummet er repræsenteret med de forskellige veje til et nyt rum. Herfra skal der hentes beskrivelser af de forskellige rum fra databasen via Back-End modulet. Derved når der skiftes rum i Front-end modulet, så sendes beskrivelsen til Front-End modulet som derefter kan vise beskrivelsen for det specifikke rum til brugeren.

\subsection{Player class og Items}
Heri skal der implementeres stats til spilleren. I vores implementation kommer der to færdigheder som kan have en indflydelse på spillerens forløb under combat. De to færdigheder er HealthPoints og ArmorClass. HealthPoints skal udgøre hvor meget skade spilleren kan tage inden død. ArmorClass skal udgøre hvor højt fjenden skal slå for at give skade til spilleren. Heri skal implementeres items som kan forøge spillerens chancer for at gennemføre de forskellige fjender. Heri vil der være 'shields' som forøger variablen ArmorClass for spilleren, så fjenden skal slå højere for at give skade til spilleren. Den anden type item er våben som er med til at øge spillerens 'terning-rul' for at slå højere end fjendens ArmorClass og skade fjenden. Combat-fasen bliver uddybet yderligere i understående afsnittet \autoref{sec:Combat-design}.

\subsection{Enemies}
Fjender skal implementeres på samme måde som spilleren. Deres færdigheder såsom HealthPoints, ArmorClass og Attack(våben)skal dog være forudbestemt. Fjenderne er fordelt over mappet og skal være gradvist stærkere jo længere spilleren kommer i spillet. Dette skal gøres ved at forøge deres færdigheder. Disse fjenders færdigheder og navne skal hentes fra GameControlleren af Front-End, når spillet startes af brugeren. Når bruger så går ind i et givet rum, hvor en fjende er til stede, går spillet ind i combat state. Når fjender er blevet bekæmpet, skal lagres det i databasen via back-end modulet når bruger vælger at gemme spillet. Når bruger dernæst åbner det gemte spil, er fjenden bekæmpet og fjernet fra spillet.
\subsection{Combat}
\label{sec:Combat-design}


\subsection{Saves}