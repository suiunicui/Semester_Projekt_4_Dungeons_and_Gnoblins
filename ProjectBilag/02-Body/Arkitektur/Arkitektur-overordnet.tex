\section{Arkitektur}

\subsection{Systemarkitektur}
\noindent
Dette afsnit forklarer hvordan systemets toplevel arkitektur er opbygget.
Denne består af en bruger som interagerer med systemet gennem frontendapplikationen som skrives i WPF. States i denne applikation styres af Game modul som holder styr på hvor spilleren befinder sig, hvilke items der er samlet op og andre nyttige information som skal bruges gennem spillet.
Spillets backend benyttes primært til bruger authentication og som bindeled til databasen.
Databasen indeholder oplysninger om blandt andet brugere, de gemte spil og oplysninger om historien for de forskellige rum.
For at se en visuel repræsentation af hvordan de forskellige moduler snakker sammen se \autoref{fig:Arkitektur-SD-SaveGame}

\begin{figure}[H]
\centering
\includegraphics[width = \textwidth]{02-Body/Images/Arkitektur - C4TopLevel.pdf}
\caption{C4 Top-Level diagram for systemets arkitektur. Her ses et diagram for systemets Top-Level arkitektur, hvori der er skabt et overblik over hvilke moduler der er til stede i systemet og hvordan de kommunikere. Heri er der også tilføjet kommunikationsmetode for de forskellige forbindelser.}
\label{fig:Arkitektur-SD-SaveGame}
\end{figure}

\subsection{Front-end Arkitektur}

Front-end applikationen er det man kan kalde, brugerens vindue til spillet, det er i dette modul at brugeren vil få alt sit information og vil få mulighed for at lave inputs til spillet. Det er yderligere her at der skal tages hånd om brugerens input således at de rigtige funktioner i Game Enginen bliver kaldt, når brugeren trykker på en vilkårlig knap.\\
Front-end'en er opbygget af et hav af forskellige skærme og menuer, og for at give et overblik over disse er der lavet følgende C3 model for Front-end'en (\autoref{fig:Arkitektur-FrontEnd-C3}), som giver en idé om hvilke skærme og menuer der kan gå til hvilke andre menuer/skærme. Udover dette fortæller modellen også om hvordan og hvilke skærme og menuer der snakker med noget uden for Front-end'en selv f.eks. skal der ved Login/Register kontaktes databasen for at få verificeret logind-oplysninger, og samtidig skal der ved save og load-game hentes en liste af gemte spil i databasen hvorefter der skal henholdsvis skrives og hentes fra databasen alt efter om man gemmer eller henter et spil.

\begin{figure}[H]
\centering
\includegraphics[width = \textwidth]{02-Body/Images/Frontend_C3.pdf}
\caption{C3-Model for Front-end. Modellen fortæller hvordan man kan navigere igennem forskellige menuer og hvilke menuer der kan føre til hvad. Derudover kan man se hvilke blokke der snakker ud af front-end'en og sammen med resten af systemet.}
\label{fig:Arkitektur-FrontEnd-C3}
\end{figure}

\subsubsection{Pseudo Front-end Arkitektur}
For at give overblik over, hvordan kommunikationen mellem frontend, backend og gamecontroller kommer til at foregå, er der lavet et pseudo sekvensdiagram for følgende UserStories:
\\
- Login\\
- Register\\
- Save Game\\
- Load Game\\

Der vil i dette afsnit kun blive vist "Save Game". "Load Game","Login" og "Register" kan findes i Tekniskbilag \textbf{(INDSÆT REFERENCE HER)}.

\noindent På \autoref{fig:Arkitektur-FrontEnd-Save} ses "Save Game", som viser forløbet når en bruger gerne vil gemme sit igangværende spil, set fra Frontends perspektiv. Her kan man bide mærke i, at når der skiftes skærm, vil den nye skærm få initialiseret sine variabler i sin constructor og derfor er der kun et selvkald, hver gang skærmen skiftes. Dette selvkald tager hånd om at opdatere mappen således at spilleren er i det rigtige rum, og at man kun kan se det af mappen, man har været i.
Hertil skal der nævnes at hvis brugeren er i "Combat State" er det ikke muligt at gemme spillet og knappen "Save Game" på "In Game Menu" vil ikke kunne ses eller bruges.\\

\begin{figure}[H]
\centering
\includegraphics[width = \textwidth]{02-Body/Images/Front-End_-_Arkitektur-savegame.pdf}
\caption{Pseudo sekvensdiagram af forløbet af userstory "Save Game", set fra Frontends perspektiv. Der laves 2 kald til databasen igennem Backenden, hvori der i det første kald,  "Hent liste af gemte saves" hentes en liste af brugerens gemte spil og i andet kald gemmes brugerens nuværende spil henover det valgte spil.}
\label{fig:Arkitektur-FrontEnd-Save}
\end{figure}

\newpage

\subsection{GameController Arkitektur}
Game Controllerns rolle er at skabe logikken for brugeren i spillet. Game Controllerens rolle for systemet er afgørende, når spillet er startet for brugeren. Game Controllerens funktionalitet indebærer blandt andet at skabe et map for brugeren, så spilleren kan flytte fra rum til rum ved start af spil. Her forekommer det at brugeren anvender Front End til at interagere med spillet og derefter kan funktionerne fra Game Controlleren kaldes.
\subsubsection{States og Character interagering}
\begin{figure}[H]
\centering
\includegraphics[width = \textwidth]{02-Body/Images/Arkitektur - State Logic.pdf}
\caption{Flow Chart over systemets states når spillet er startet. Dette bilag viser arkitekturen over det ønskede states i systemet når bruger har startet spillet. Dette viser også hvordan spillets fremgang er ønsket og hvordan bruger kommer videre i spillet igennem de forskellige states.}
\label{fig:Arkitektur-SD-SaveGame}
\end{figure}
\subsubsection{Room State og Character interagering}
Det andet som Game controlleren har ansvaret for, er spillerens våben og skjold. Game Controlleren er delt op i spillets game states som kan ses i bilaget ovenover. Der er hovedsageligt 2 states når spillet er startet. Room State og Combat state. Når spillet er startet for brugeren, starter brugeren i et room state. I dette state kan der interageres med rummet, hvis der er genstande til stede. Her kan brugeren også brugerdefinere sin spiller ved brug af knappen inventory og skifte våben eller skjold. Her kan der også se spillerens evner ved brug af knappen Character. Derudover er der også implementeret beskrivelser fra de forskellige rum som hentes fra modulet Database via modulet Back end.
\subsubsection{Combat State og Combat simulering}
Combat state indebærer når spilleren møder en fjende. Dette sker når man går ind i forskellige rum. Der er oprettet klasser for spillerens evner i form af Attack(Spillerens evne til at slå) og Armor(Spillerens evne til at modstå angreb). Når spilleren angriber en fjende foregår det ved at brugeren trykker på knappen ’Attack’. Når dette forekommer er der implementeret en terning i Game Controlleren. Dette skal implementeres ved to implementeringer. En simuleret terning i form af en pseudo random-number generator som genererer et tal mellem 1 og parameteren numOfSides og en  En Rekursive Psudo-Random number generator som tager en tuple (numOfSides, numOfDice). Den gentager rekursivt implementering 1. Et antal gange svarende til NumOfDice parameteren og summere alle resultaterne.  Hvis brugeren taber kampen, skal brugeren starte et nyt spil eller loade et save. I figuren under ses et Flow chart over forløbet når spiller går ind i combat state.
\begin{figure}[H]
\centering
\includegraphics[width = \textwidth]{02-Body/Images/Arkitektur - Combat State.pdf}
\caption{Flow Chart over spillets fremgang når spiller går ind i combat State. Når spiller går ind i et rum med en fjende i rummet, går spillet ind i combat state. Herfra skal spillets terning simulerer et tilfældigt nummer ud fra spilllerens evner. Fjendens evner er bestemt i forvejen. Angreb fastlægges om spiller ruller højere end fjendens Armor Class og vice versa De tre outcomes for spilleren er følgende: Spiller flygter(flee-knap), Spilleren mister alt liv(Spil sluttes) og fjende mister alt liv(Spiller går til room state)}
\label{fig:Arkitektur-SD-SaveGame}
\end{figure}
\subsubsection{Forbindelser til andre moduler}
Game Controlleren er forbundet med systemets Back-end og Front-End. Den håndterer data fra Back-end som derefter kan håndteres, så Front-end kan kalde funktionerne fra Game Controlleren. Det data som skal håndteres er primært gemmets forløb som kan lagres i databasen under brugeren. Herefter kan bruger loade det samme gem igen med samme fremgang som da bruger gemte spillet. Det vil være nuværende rum, rum der er blevet besøgt, fjender der er blevet bekæmpet og genstande der er samlet op og taget på.
\subsection{Backend Arkitektur}
\label{ssec: BackendArkitektur}
I dette afsnit gives en redegørelse for backendens arkitektur samt formål.\\ 
 
\noindent Backend containeren består overordnet set af to dele et Web Api og en database, i dette afsnit vil fokusset ligge på Web Api'et for flere detaljer omkring databasens arkitektur henvises til afsnit \textbf{Mangler ref til db ark}. Web Api'et vil blive udviklet i frameworket ASP.NET Core. Web api'et skal kunne tilgåes af clienter igennem HTTP request/responses, for at hente/sende gemte spil. Web api'et kontakter efterfølgende databasen igennem et DAL, som udfører de nødvendige queries. Hertil skal Web api'et sørge for Authentication og Authorization den bruger som er logget ind på clienten hvorfra der kaldes. \\

\noindent Til at beskrive backendens arkitektur udarbejdes et C4 level 3 diagram. Diagrammet som kan ses på \autoref{fig:Arkitektur-Backend-C3} giver et overblik over hvilke componenter backenden består af, hvilke teknologier det forventes at anvende, samt hvordan de kommunikerer indbyrdes og med resten af systemet.\\

 
\begin{figure}[H]
\centering
\includegraphics[width = \textwidth]{02-Body/Images/Backend_C3.PNG}
\caption{C3-Model over Backend container,figuren giver et overblik over de componenter backenden består, samt hvordan disse kommunikere indbyrdes og med omkring liggende containere.}
\label{fig:Arkitektur-Backend-C3}
\end{figure}

\noindent Figuren illustrerer en lagdelt struktur, som gør brug af MVC mønstret. Der ses en client (Game Module) som kalder ned i to controllere Authentication og Save Controller. Kommunikationen her i mellem foregår med HTTPS req/resp, med dataen på JSON format.\\

\noindent \textbf{Controller:}\\
Controllerne opdeles efter ansvar såldes der er en som står for bruger håndtering og en som står for at gemme/hente spil. Disse vil bestå af ASP.Net MVC Controllere \cite{MVC controller}.

\noindent \textbf{DAL:}\\
De to controllere kalder videre ned i DAL laget, som består af en database kontekst og en klasse til at håndtere queries for SaveControlleren. Til at skabe denne database kontekst anvendes EF Core \cite{EF Core}.

\noindent \textbf{Model:}\\
Modellerne fungerer som bindeledet mellem alle komponenterne, de definerer den data som der arbejdes på, og vil bestå af en række C\# klasser.\\

\noindent MVC mønstret bidrager med at skabe højere samhørighed i applikationen ved at muliggøre en logisk opdelling af funktionalitet i controllerne, ved at de tager ansvar for hver deres områder af kommunikationen (Bruger og spil). Dets ses også at koblingen mellem modulerne er forholdsvis lav såldes, således det er let at bygge videre på og tilføje ny funktionalitet uden at ødelægge noget.\\

\subsubsection{REST}
I designet af Web Api'et søges det at opnå følgende 5 REST principper \cite{REST}.

\begin{enumerate}
 \item Alle de data objekter der arbejdes med tilhører en bestemt unique URI. Dataen hentes, sendes og manipuleres igennem standard HTTP metoder som (GET, POST, PUT, DELETE).
 \item Der arbejdes ud fra en client/server arkitektur med data som resource. 
 \item Web Api’et er stateless, hvilket vil sige der gemmes ikke nogen tilstand omkring clienten på server siden.
 \item Der arbejdes ud fra en lagdelt struktur, som betyder at hver component kun kan se de componenter som grænser op til den selv.
 \item Der anvendes Cashing på server siden (Dette vil ikke blive implementeret).
\end{enumerate}

\noindent Dette bidrager til at data overførelserne ikke bliver for komplekse og for at overholde SoC, ved at adskille repræsentationen og dataen fra hinanden.

\subsubsection{Konlusion}

Systemets backend består at Web Api, som bygges op omkring MVC mønstret, Web Api’et vil gøre brug af REST principper for at adskille data resourcerne fra deres repræsentation på clientens side.

\newpage

\subsection{Database Arkitektur}

I oprettelsen af systemets database skal der tages hånd om, hvordan kommunikationen skal foregå imellem systemets segmenter, samt databasens funktionalitet. Her er der blevet besluttet at der anvendes en DAL, der fungerer ved at hver gang databasen skal kontaktes, så foregår det igennem denne. Yderligere vil denne DAL også simplificere kommunikationen mellem databasen og Back-End.

\begin{figure}[H]
\centering
\includegraphics[width = \textwidth]{02-Body/Images/C4TopLvlDB}
\caption{C4 top level diagram, som viser kommunikation mellem systemets segmenter}
\label{fig:C4TopLvlDB}
\end{figure}

Måden kommunikationen vil foregå igennem systemet vises her. Her ses der at en bruger interagere med Front-End, data går videre til Game Module, som kommunikere med Back-End, hvor der til sidst enten bliver skrevet til databasen eller hentet data fra databasen
For at illustrere databasens funktionalitet er der blevet dannet sekvensdiagrammer, som viser hvordan databasen vil kunne gemme et spil og hente et gemt spil.

For at illustrere databasens funktionalitet er der blevet dannet sekvensdiagrammer, som viser hvordan databasen vil kunne gemme et spil og hente et gemt spil.

\begin{figure}[H]
\centering
\includegraphics[width = \textwidth]{02-Body/Images/RoomDescriptionsDB.PNG}
\caption{Sekvensdiagram for hvordan rum beskrivelser vil blive hente fra databasen af SaveController}
\label{fig:RoomDescriptionsDB}
\end{figure}

I diagrammet GetRoomDescription ses der, hvordan kommunikationen ville foregå for at hente rum beskrivelsen. Her ses der at SaveController, fra Back-End, går til DAL, som så henter rum beskrivelsen fra databasen, og herefter returneres dataene fra databasen til DAL og så fra DAL til SaveController.

\begin{figure}[H]
\centering
\includegraphics[width = \textwidth]{02-Body/Images/GetAllSavesDB.PNG}
\caption{Sekvensdiagram for User story 17: GetAllSaves, i relation til hentet data fra database}
\label{fig:GetAllSavesDB}
\end{figure}

I diagrammet GetAllSaves foregår kommunikation på samme måde som ved at hente rum beskrivelser. Forskellen her er dog at der i stedet for hentes en list af alle gemte spil.

\begin{figure}[H]
\centering
\includegraphics[width = \textwidth]{02-Body/Images/GetSaveDB.PNG}
\caption{Sekvensdiagram for User story 18: GetAllSaves, i relation til hentet data fra specifikt gemt spil fra database}
\label{fig:GetSaveDB}
\end{figure}

I diagrammet GetSave ses der hvordan et specifikt gemt spil hentes. Her ses der at det igen går først igennem DAL. Et gemt spil vil have et ID, som der kan anvendes til at finde de korresponderende værdier som dette gemte spil har. Her ses der at der vil findes items, enemies, puzzles og hvilke rum en spiller har besøgt. Dataene returneres herefter til DAL og så til SaveController.

\begin{figure}[H]
\centering
\includegraphics[width = \textwidth]{02-Body/Images/SaveGameDB.pdf}
\caption{Sekvensdiagram for User story 7: SaveGame. Dette diagram viser hvordan et spil gemmes}
\label{fig:SaveGameDB}
\end{figure}

I dette diagram, SaveGame, ses der hvordan et spil vil gemmes. Måden dette vil fungere på er at spilleren vil starte med fem tomme gemte spil, som bliver seeded til databasen. Dette bliver gjort for at begrænse en bruger til fem gemte spil. Så for at gemme et spil findes der først hvilket gemte spil der skal overskrives. Derefter findes de relevante værdier i dette gemte spil. Hernæst slettes disse værdier og der tilføjes de nye korrekte værdier.



\subsection{DAL Arkitektur}
I dette segment vil der blive forklaret tankerne bag arkitekturen vedr. oprettelsen af en funktionel database, som kan anvendes til at opbevaring af data til dette projekt.
Til oprettelsen af en funktionsdygtig database, i dette projekt, kræves der et relativt tæt sammenhold mellem databasen og backend api’en. API’en er ansvarlig for kommunikation mellem Front-end, Game Module og databasen.

\begin{figure}[H]
\centering
\includegraphics[width = \textwidth]{02-Body/Images/DALArkitektur}
\caption{C3 diagram for blokkene involveret i DAL's funktionalitet}
\label{fig:DALArkitektur}
\end{figure}


Når data enten skal sendes til eller hentes fra databasen så er det igennem et DAL. Systemets DAL giver os en simplificeret adgang til dataene gemt i databasen, og fungerer som en mellemmand for systemet, da alt kommunikation til databasen går igennem den.
I denne DAL har vi en Authentication, som bliver anvendt når en bruger logger ind, og når der oprettes en ny bruger. Denne blok kontrollerer brugernavn og kodeord, som sendes og hentes i databasen. Ydermere vil der ikke sættes fokus på sikkerhed, som ses i kravene stillet for systemet.
Yderligere vil det også være muligt at både gemme et spil og indlæse et spil. Begge af disse vil operere på det samme data, dog ville den ene, LoadGame, hente data fra systemets database, og den anden, SaveGame, vil sende data til systemets database. 
LoadGame i DAL er ansvarlig for at hente spillerens data fra databasen, såsom hvilket rum de var i og mængden af liv de har tilbage. 
Hernæst er der SaveGame. Denne indeholder funktionaliteten for at sende et gemt spil, altså dataene for spillerens nuværende spil. Heri vil der også gemmes information og spillerens nuværende tilstand i spillet. Det ville f.eks. være rummet som spilleren er i når spillet bliver gemt.
Begge af disse vil være ansvarlige for at håndtere data som er individuelle for hver spiller og hvert gemt spil.
Udenfor systemets DAL ville der også være inkluderet Models i konstruktionen af systemets database. Models vil indeholde de entities som Authentication og Load-/SaveGame vil bestå af. Yderligere vil Models også være ansvarlig for forholdene imellem de forskellige entities. 



\begin{figure}[H]
\centering
\includegraphics[width = \textwidth]{02-Body/Images/ER-RoomDescription.PNG}
\caption{ER diagram for Roomdescription. En beskrivelse består blot af en beskrivende string samt det tilhørende unikke rumid.}
\label{fig:ER-Roomdescription}
\end{figure}

