\subsection{Frontend Design}

Spillet vil bestå af en række af vinduer, som giver spilleren den nødvændige information for at de kan spille spillet (så som at logge ind, gemme og hente spil, samt spille spillet). Her følger en række af mockups af nogle vinduer.

\subsubsection{Room}

Room view (\autoref{fig:Design-FE-mockup-room}) er det primære spil-vindue. Her præsenteres spilleren for en beskrivelse af det rum de er i, samt hvilke elementer i rummet de kan interagere med. Der vises også et kort over banen. Kortet Viser kun de rum spilleren allerede har været i, mens resten holdes skjult.\\
En række af knapper nederst i højre hjørne giver spilleren mulighed for at interagere med spillet. Fire knapper ("Go {North/West/South/East}") lader spilleren gå fra et rum til et andet. Ikke alle rum har forbingelse til alle sider, så det er f.eks. ikke altid muligt at trykke på "Go North". Kortet og rum beskrivelsen fortæller hvilken vej det er muligt at bevæge sig.\\
Udover de fire retnings knapper er der et antal andre knapper. Disse bruges til at gemme spillet, gå til menuer, samt interagere med elementerne i rummet. Det specifikke antal og deres funktion er afhængig af den præcise implementering.

\begin{figure}[h]
\centering
\includegraphics[width = \textwidth]{02-Body/Images/RoomMockup.PNG}
\caption{Et mockup af det primære spil vindue. Tekst i venstre side af skærmen giver en beskrivelse af det rum spilleren er i, samt en liste af elementer i rommet som spilleren kan interagere med. Øverst til højre vises et billede af banen. Spilleren interagerer med spillet via knapper i højre hjørne. Knapperne "Go {North/West/South/East}" fører spilleren ind i et andet rum mens de resterende knapper (markeret "Button") bruges til andre funktionaliterer i spillet.}
\label{fig:Design-FE-mockup-room}
\end{figure}

\subsubsection{Combat}

Combat vinduet er baseret på room vinduet. Strukturen er den samme: der er et kort øverst til højre, en beskrivelse øverst til venstre og knapper neders til højre. Nederst til venstre er der inform om hvordan kampen går,  i stedet for en liste af elementer i rummet.\\
Knapperne består af nogle "menu" knapper, som lader dig gå til spil menuer, samt en 'Fight' knap og en 'Flee' knap. 'Fight' knappen lader spilleren kæmpe mod fjenden, mens 'Flee' knappen lader spilleren flygte fra kampen.


\begin{figure}[h]
\centering
\includegraphics[width = \textwidth]{02-Body/Images/CombatMockup.PNG}
\caption{Combat view. Spilleren præsenteres for en fjende, og får information om hvordan kampen mod fjenden går. Der er knapper til at kæmpe og flygte, samt gå til menuer. øverst til højre er kortet over banen, ligesom i Room vinduet \autoref{fig:Design-FE-mockup-room}.}
\label{fig:Design-FE-mockup-combat}
\end{figure}


\subsubsection{Login}

Når spillet starter bedes spilleren logge ind på deres profil. Dette sker i login vinduet. Spilleren kan indtaste sit brugernavn og kodeord i de to tekstfelter 'Username' og 'Password'. Knappen login fører dem videre til spillet, hvis det indtastede login er korrekt.\\
knappen create user fører spilleren til et vindue som ligner login vinduet, og som lader spilleren oprætte en ny bruger.\\
Exit lukker spillet.

\begin{figure}[h]
\centering
\includegraphics[width = \textwidth]{02-Body/Images/LoginMockup.PNG}
\caption{Login view. Bruger kan indtaste sit brugernavn og kodeord for at få adgang til spillet, eller trykke på create user for at komme til et vindue hvor man kan oprætte en ny bruger. Det er også muligt at forlade spillet igen.}
\label{fig:Design-FE-mockup-login}
\end{figure}

\subsubsection{Settings}
Settings viduet tillader at spilleren kan ændre indstillinger for spillet, og kan tilgås fra hovedmenuen, samt fra selve spillet.\\
Det er her muligt at ændre f.eks. skærmopløsning og lydstyrke. Det er muligt at gemme de indstillinger som er valgt, forlade menuen uden at gemme de valgte indstillinger, samt gendanne standard indstillingerne for spillet. 

\begin{figure}[h]
\centering
\includegraphics[width = \textwidth]{02-Body/Images/settingsMockup.PNG}
\caption{Settings view. Tillader at man kan ændre indstillinger for spillet, så som skærmopløsning og lydstyrke. Det er muligt at gemme indstillingerne, forlade skærmen uden at gemme indstillingerne, samt sætte spillet tilbage til standard indstilinger.}
\label{fig:Design-FE-mockup-settings}
\end{figure}

\newpage
