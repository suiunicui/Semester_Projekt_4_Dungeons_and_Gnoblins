\section{Metode og Proces}
I nedstående afsnit fremlægges metoder og processer brugt 
til udviklingen af Dungeons and Gnoblins spillet. 


\subsection{UML}

\subsubsection{Gruppedannelse}
Gruppen blev dannet på baggrund af at vi som studerende selv skulle finde medlemmer og danne projektgrupper. 4 af gruppens medlemmer var dannet inden semesterets start og de resterende 4 blev indmeldt efter offentliggørelsen af semesterprojektet. Gruppen redegjorde først ideer til projektet og blev hurtigt forventningsafstemt om projektets resultat skulle være middelmådigt. 

\subsubsection{Projektforløb og møder}
Projektets endelige mål har været implementeringen af et text-based adventure game.
Hertil har gruppen gjort brug af agile processen til at imødekomme dette mål.

Først har gruppen specifiseret Kravspecifikationerne for spillet, med ønsker om
hvilke features spillet har skulle inkludere. Continuous integration er dernæst
været et vigtigt værktøj til implementeringen af de ønskede features, med et klart
ønske om at integrere nye features ind i projektet få tidligt som muligt.\\

I løbet af projektforløbet blev der afholdt to faste ugentlige møder. Et internt gruppemøde om mandagen og vejledermøde om onsdagen. Hertil var der fra start uddelt rollerne Mødeleder og referent til gruppens medlemmer som gik på skift hver uge. Til hvert møde blev der gennemgået hvad der var blev arbejdet med, hvilke udfordringer der forekom samt hvad der fremadrettet skulle arbejdes med for hver af gruppens medlemmer. Dette gjorde vi i form af SCRUM for at give et bedre overblik for projektets fremgang og dermed vurderer om gruppen var bagud eller foran ift. Gruppens tidsplan. Vores SCRUM-proces var ikke avanceret hvor der blev koordineret SCRUM Master og Product Owner, men blev derimod anvendt som redskab til at skabe et overblik over arbejdsfordelingen og udviklingen. Tidsplanen blev udarbejdet ved projektets start og havde formålet at give gruppen deadlines for projektets udviklingsproces.

\begin{figure}[H]
  \centering
  \includegraphics[scale=.5]{02-Body/Images/Agile.png}
  \caption{Viser et billed af agile modellen som består af mangle iterationer
           af design, develop, test, og deploy. Dette minder om continuous integration
           og har hindret mange problemmer for udviklingsforløbet}
  \label{fig:Agile}
\end{figure}

SCRUM og AGILE bringer klarhed til medlemmerne om deres roller og opgaver over en 
kommende tidperiode med en backlog over opgaver, som skal færdiggøres over et sprint (1 uge).
Ugentlige opdateringer og møder omkring potentielle problemmer udviklingsforløbet betyder
at gruppen har kunne tage hånd om evt. problemmer tidligt i forløbet og derved løse dem
før de har udviklet sig til større problemmer.

Til styring og implementering af souce code har vi benytte git som et version control system.
fordelen ved git er at det tillade flere udviklere at arbejde på det samme projekt og integere 
deres løsninger. Der er dog chancer for merge konflikter hvilket kan, i værkste tilfælde, før
til problemmer med at integrer de forskellige løsninger korrekt med hinanden.

\subsection{Modellering}

Projektet benytter UML til at beskrive og modellere software-- arkitektur
og design. UML giver et nemt overblik over projektets struktur, og den 
ønskede opførsel for programmets forskellige features.

\textbf{(IMAGE HER)}

STM og SD diagrammer er brugt til at give programmets udførsel struktur, og fungere
som en hjælp til at visualisere programmets forskellige states og flow of execution.
Det giver et nemt overblik over hvordan funktionskald mellem forskellige komponenter
har skal fungere således, at der ikke opstår misforsåelse grupperne imellem. 

Generelt er arkitekturen opbygget som "pseudo" diagrammer. Dvs. de ligner ikke fuldstændigt 
det endelige design, men har til formål at vise den overordnede tankegang i systemet og 
samtidig virke som et udviklingsværktøj til at videreudvikle systemet. Da der i gruppen 
er blevet arbejdet iterativt, er der forskelle mellem diagrammer for arkitektur og design,
da der er blevet tilføjet flere moduler og funktioner undervejs. Den overordnede tanke i 
projektet er der dog ikke blevet ændret på.

En C4 model fungere som et high level visualisering af hele systemets arkitektur.
Her kan ses hvilket moduler, der er i stand til at kommuniker med hinanden og hvilke,
der inteagere med brugeren.

\subsection{Udviklingsforløb}
Projekts ultimative formål har været at implementere et text-based adventur game. 
Forløbet mod Implementeringen af dette mål har været en iterative process hvor 
der først er lavet et udkast til den overordnede C4 arkitektur samt et udkast
til Kravspecifikationerne, der tjener som dokumentation for de ønskede features 
spillet skal implemntere før det er færdigt.

\subsubsection{Iterativ Udviklingsforløb}

Efter Kravspecifikationerne er der opstået en iterative proces hvor person er
blevet tildelt et hovedansvarsområde som en del af en gruppe. Hver uge rapportere
hver ansvarsområde hvor langt de noget samt deres mål for den forestående uges arbejde.

Tidligt i projektet har der været forsøg på at lave continous integration så alle moduler
kun kommunikere sammen tidligt i processen. Det har ført til at hver nye feature derefter 
har været nem at integrere, som det sidste led i hver iteration efter færdiggørelsen af en
ny feature. 

Denne flexible arbejdsmetode har igen ført til at problemmer ikke har kunne vokse men
at der blevet taget hånd om dem mens det stadig har været muligt at håndtere dem.

