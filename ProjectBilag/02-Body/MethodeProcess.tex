\section{Metode og Proces}
I nedstående afsnit fremlægges metoder og processer brugt 
til udviklingen af Dungeons and Gnoblins spillet. 

\subsection{UML}
Projektet benytter UML til at beskrive og modellere software-- arkitektur
og design. UML giver et nemt overblik over projektets struktur, og den 
ønskede opførsel for programmets forskellige features.

\textbf{(IMAGE HER)}

STM og SD diagrammer er brugt til at give programmets udførsel struktur, og fungere
som en hjælp til at visualisere programmets forskellige states og flow of execution.

En C4 model fungere som et high level visualisering af hele systemets arkitektur.
Her kan ses hvilket moduler, der er i stand til at kommuniker med hinanden og hvilke,
der inteagere med brugeren.

\subsection{Udviklingsforløb}
Projekts ultimative formål har været at implementere et text-based adventur game. 
Forløbet mod Implementeringen af dette mål har været en iterative process hvor 
der først er lavet et udkast til den overordnede C4 arkitektur samt et udkast
til Kravspecifikationerne, der tjener som dokumentation for de ønskede features 
spillet skal implemntere før det er færdigt.

\subsubsection{Iterativ Udviklingsforløb}

Efter Kravspecifikationerne er der opstået en iterative proces hvor person er
blevet tildelt et hovedansvarsområde som en del af en gruppe. Hver uge rapportere
hver ansvarsområde hvor langt de noget samt deres mål for den forestående uges arbejde.

Tidligt i projektet har der været forsøg på at lave continous integration så alle moduler
kun kommunikere sammen tidligt i processen. Det har ført til at hver nye feature derefter 
har været nem at integrere, som det sidste led i hver iteration efter færdiggørelsen af en
ny feature. 

